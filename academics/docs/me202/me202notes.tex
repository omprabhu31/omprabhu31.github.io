\documentclass[11pt]{article}
\usepackage[utf8]{inputenc}	

%common math and LaTeX packages
\usepackage{amsmath,amsthm,amsfonts,amssymb,amscd}
\usepackage{multirow,booktabs}
\usepackage[table]{xcolor}
\usepackage{multirow}
\usepackage{fullpage}
\usepackage{lastpage}
\usepackage{enumitem}
\usepackage{fancyhdr}
\usepackage{mathrsfs}
\usepackage{wrapfig}
\usepackage{setspace}
\usepackage{calc}
\usepackage{multicol}
\usepackage{cancel}
\usepackage[retainorgcmds]{IEEEtrantools}
\usepackage[margin=3cm]{geometry}
\usepackage{amsmath}
\newlength{\tabcont}
\setlength{\parindent}{0.0in}
\setlength{\parskip}{0.0in}
\usepackage{empheq}

%shaded environment for important equations/notes
\usepackage{mdframed}
\colorlet{shaded}{blue!15}
\colorlet{shadedtext}{black}
\newenvironment{shaded}
   {
     \raggedright
     \color{shadedtext}%
   }{}
\surroundwithmdframed[
   hidealllines=true,
   backgroundcolor=shaded
]{shaded}

%page geometry definitions
\usepackage[most]{tcolorbox}
\usepackage{xcolor}
\parindent 0in
\parskip 6pt
\geometry{margin=1in, headsep=0.25in}

%custom theorem definitions
\theoremstyle{definition}
\newtheorem{innercustomgeneric}{\customgenericname}
\providecommand{\customgenericname}{}
\newcommand{\newcustomtheorem}[2]{%
  \newenvironment{#1}[1]
  {%
   \renewcommand\customgenericname{#2}%
   \renewcommand\theinnercustomgeneric{##1}%
   \innercustomgeneric
  }
  {\endinnercustomgeneric}
}
\newcustomtheorem{thm}{Theorem}
\newcustomtheorem{lem}{Lemma}
\newcustomtheorem{defn}{Definition}
\newcustomtheorem{prop}{Proposition}
\newcustomtheorem{exer}{Exercise}
\newcustomtheorem{note}{Note}
\renewcommand{\qedsymbol}{$\blacksquare$}

\let\a\alpha
\let\b\beta
\let\g\gamma
\let\e\varepsilon
\let\t\theta
\newcommand{\R}{\mathbb{R}}
\newcommand{\Q}{\mathbb{Q}}
\newcommand{\Z}{\mathbb{Z}}
\newcommand{\N}{\mathbb{N}}
\newcommand{\PP}{\mathcal{P}}
\newcommand{\C}{\mathcal{C}}
\newcommand{\Lagr}{\mathcal{L}}

\begin{document}

%document header
\begin{center}
{\LARGE \bf ME 202 - Strength of Materials}\\
{Instructor: \textit{Prof. Salil Kulkarni}}\\
Last updated \today \\~\\
{\large \bf Om Prabhu}\\
Undergraduate, Mechanical Engineering\\
Indian Institute of Technology Bombay\\~\\
\textsc{Disclaimer}
\end{center}
\vspace{-6pt}

This document is a compilation of the notes I made while taking the course ME 202 (Strength of Materials) in my 4$^{\text{th}}$ semester at IIT Bombay. It is not meant to serve as a replacement for any formal textbook or lecture on the subject, since the theory is not discussed at all.

There will probably be many instances where I use certain common symbols without explicitly mentioning what they mean. It is to be assumed that they carry their usual meanings.

If you have any suggestions and/or spot any errors, you know where to contact me.

\hrulefill
\tableofcontents
\hrulefill
\pagebreak
\section{Torsion of Circular Shafts}
\begin{center}
\begin{tabular}{p{0.1\textwidth}p{0.3\textwidth}p{0.1\textwidth}p{0.2\textwidth}l}
\cline{1-4}
\multicolumn{1}{|l|}{rod}   & \multicolumn{1}{l|}{axial loading}            & \multicolumn{1}{l|}{\multirow{2}{*}{frame}} & \multicolumn{1}{l|}{\multirow{2}{*}{\begin{tabular}[c]{@{}l@{}}axial as well as\\ shear loading\end{tabular}}} &  \\ \cline{1-2}
\multicolumn{1}{|l|}{beam}  & \multicolumn{1}{l|}{transverse/shear loading} & \multicolumn{1}{l|}{}                       & \multicolumn{1}{l|}{}                                                                                          &  \\ \cline{1-4}
\multicolumn{1}{|l|}{shaft} & \multicolumn{1}{l|}{torsional loading}        &                                             &                                                                                                                &  \\ \cline{1-2}
                            &                                               &                                             &                                                                                                                & 
\end{tabular}
\end{center}
\vspace{-12mm}
\begin{itemize}
	\itemsep-0.25em
	\item[$-$] torque: causes twist or \textit{torsion} in a machine element
	\item[$-$] shaft: transmits rotary motion from one location to another
\end{itemize}
\vspace{-7mm}
\begin{enumerate}[label=\textbf{\roman*)}]
	\item \textbf{Internal resisting torque} (method of sections)
\end{enumerate}
\vspace{-8mm}

\begin{wrapfigure}[6]{L}{0.42\textwidth}
\includegraphics[width=0.4\textwidth]{section_analysis.JPG}
\end{wrapfigure}

\hspace{0.5\textwidth}
\vspace{-1.5mm}

FBDs at different sections:
\vspace{2.5mm}

\includegraphics[width=0.5\textwidth]{section_FBD.JPG}

$\sum M_{AB}=0\implies -250+T_{AB}=0$

$\sum M_{BC}=0\implies -250+75+T_{BC}=0$

$\sum M_{CD}=0\implies -250+75+325+T_{CD}=0$

Direction of torque is determined using right hand thumb rule, i.e. thumb along +ve z-direction and direction of curling of fingers corresponds to +ve torque.
\end{document}