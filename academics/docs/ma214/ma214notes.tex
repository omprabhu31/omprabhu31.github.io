\documentclass[11pt]{article}
\usepackage[utf8]{inputenc}	

%common math and LaTeX packages
\usepackage{amsmath,amsthm,amsfonts,amssymb,amscd}
\usepackage{multirow,booktabs}
\usepackage[table]{xcolor}
\usepackage{fullpage}
\usepackage{lastpage}
\usepackage{enumitem}
\usepackage{fancyhdr}
\usepackage{mathrsfs}
\usepackage{wrapfig}
\usepackage{setspace}
\usepackage{calc}
\usepackage{multicol}
\usepackage{cancel}
\usepackage[retainorgcmds]{IEEEtrantools}
\usepackage[margin=3cm]{geometry}
\usepackage{amsmath}
\newlength{\tabcont}
\setlength{\parindent}{0.0in}
\setlength{\parskip}{0.0in}
\usepackage{empheq}

%shaded environment for important equations/notes
\usepackage{mdframed}
\colorlet{shaded}{blue!15}
\colorlet{shadedtext}{black}
\newenvironment{shaded}
   {
     \raggedright
     \color{shadedtext}%
   }{}
\surroundwithmdframed[
   hidealllines=true,
   backgroundcolor=shaded
]{shaded}

%page geometry definitions
\usepackage[most]{tcolorbox}
\usepackage{xcolor}
\parindent 0in
\parskip 6pt
\geometry{margin=1in, headsep=0.25in}

%custom theorem definitions
\theoremstyle{definition}
\newtheorem{innercustomgeneric}{\customgenericname}
\providecommand{\customgenericname}{}
\newcommand{\newcustomtheorem}[2]{%
  \newenvironment{#1}[1]
  {%
   \renewcommand\customgenericname{#2}%
   \renewcommand\theinnercustomgeneric{##1}%
   \innercustomgeneric
  }
  {\endinnercustomgeneric}
}
\newcustomtheorem{thm}{Theorem}
\newcustomtheorem{lem}{Lemma}
\newcustomtheorem{defn}{Definition}
\newcustomtheorem{prop}{Proposition}
\newcustomtheorem{exer}{Exercise}
\newcustomtheorem{note}{Note}
\renewcommand{\qedsymbol}{$\blacksquare$}

\let\a\alpha
\let\b\beta
\let\g\gamma
\let\e\varepsilon
\let\t\theta
\newcommand{\R}{\mathbb{R}}
\newcommand{\Q}{\mathbb{Q}}
\newcommand{\Z}{\mathbb{Z}}
\newcommand{\N}{\mathbb{N}}
\newcommand{\PP}{\mathcal{P}}
\newcommand{\C}{\mathcal{C}}
\newcommand{\Lagr}{\mathcal{L}}

\begin{document}

%document header
\begin{center}
{\LARGE \bf MA 214 - Introduction to Numerical Analysis}\\
{Instructor: \textit{Prof. Saikat Mazumdar}}\\
Last updated \today \\~\\
{\large \bf Om Prabhu}\\
Undergraduate, Mechanical Engineering\\
Indian Institute of Technology Bombay\\~\\
\textsc{Disclaimer}
\end{center}
\vspace{-6pt}

This document is a compilation of the notes I made while taking the course MA 214 (Introduction to Numerical Analysis) in my 4$^{\text{th}}$ semester at IIT Bombay. It is not meant to serve as a replacement for any formal textbook or lecture on the subject, since the theory is not discussed at all.

There will probably be many instances where I use certain common symbols without explicitly mentioning what they mean. It is to be assumed that they carry their usual meanings.

If you have any suggestions and/or spot any errors, you know where to contact me.

\hrulefill
\tableofcontents
\hrulefill
\pagebreak
\section{Interpolation Theory}
Suppose $(n+1)$ real points $(x_0,y_0), (x_1,y_1),\dots,(x_n,y_n)$ are known. Further the set of points $x_i$ is spread out over the interval $[a,b]$. Then the problem of approximating a function over the interval $[a,b]$ passing through these points is called interpolation.

We define the norm on $\mathcal{C}[a.b]$ as:
$||f||=\max_{x\in[a,b]}|f(x)|$. To define the `closeness' of 2 functions formally, we consider the quantity
$||f-g||=\max_{x\in [a,b]}\left|f(x)-g(x)\right|$. The Weierstrass approximation theorem states that:

Take a function $f\in\C[a,b]$. Given any real number $\e>0$, there exists a polynomial $p$ such that 
$$||f-p||<\e\implies |f(x)-p(x)|<\e\text{\hspace{12pt}}\forall x\in[a,b]$$
\subsection{Lagrange interpolation formula} 
Given $n+1$ distinct real points $x_0,x_1,\dots,x_n$ and $n+1$ real numbers $y_0,y_1,\dots,y_n$, there exists a unique polynomial $p_n\in\mathbb{P}_n$ such that $p(x_i)=y_i$ for $ i=0,1,\dots,n$. Construct $n^{th}$ degree polynomials $L_0^n(x),L_1^n(x),\dots,L_n^n(x)$ such that
$$L_k^n(x_i)=\left\lbrace\begin{array}{lr}
1 & \text{if } i=k\\
0 & \text{if } i\neq k
\end{array}\right.\implies \boxed{p_n(x)=\sum_{k=0}^ny_kL_k^n(x)}$$
The lagrange polynomials $L_k^n$ can be found as 
$$\boxed{L_k^n(x)=\prod_{j=0,j\neq k}^n\frac{(x-x_j)}{(x_k-x_j)}}$$

\subsection{Newton's divided differences}
Let $x_0,x_1,\dots,x_n$ be $n+1$ real distinct points in $[a,b].$ Let $f:[a,b]\to\R$ be a function whose values are known at these points. We want to find a polynomial $p_n(x)\in\mathbb{P}_n$ such that $p_n(x_i)=f(x_i)$ for $i=0,1,\dots,n$.

We define the divided differences (independant of order of points) as follows: $$f[x_0]:=f(x_0)$$$$
f[x_0,x_1,\dots,x_{m+1}]:=\frac{f[x_1,\dots,x_{m+1}]-f[x_0,\dots,x_m]}{x_{m+1}-x_0}$$
Then the polynomial $p_n(x)$ can be written as:
$$p_n(x) := f[x_0]  + f[x_0, x_1](x - x_0)+\cdots+f[x_0,x_1,\dots,x_n]\prod_{k=0}^{n-1}(x-x_k)$$

\subsection{Matrix representation}
The problem can also be expressed as a system of linear equations and solved for the coefficients using matrix equations. A matrix similar to the Vandermonde matrix is generated.

$$\begin{bmatrix}
1 & x_0 & x_0^2 & \cdots & x_0^n\\
1 & x_1 & x_1^2 & \cdots & x_1^n\\
\vdots & & \ddots & & \vdots\\
1 & x_n & x_n^2 & \cdots & x_n^n\\
\end{bmatrix}
\begin{bmatrix}
a_0\\
a_1\\
\vdots\\
a_n
\end{bmatrix}=
\begin{bmatrix}
y_0\\
y_1\\
\vdots\\
y_n
\end{bmatrix}
$$
\subsection{Error estimation}
Take $f\in\C^{n+1}[a,b]$. Let $x_0,x_1,\dots,x_n$ be $n+1$ distinct points in $[a,b]$. Let $p\in\mathbb{P}_n$ such that $p(x_i)=f(x_i)$ for $i=1,2,\dots,n$. Then for all $x\in[a,b]$, there exists $\xi=\xi(x)\in (a,b)$ such that
$$f(x)-p(x)=\frac{1}{(n+1)!}f^{(n+1)}(\xi)\prod_{k=0}^n (x-x_k)$$
Taking maximum over $x\in[a,b]$, we can see that our choice of interpolation points influences the error significantly.
$$\max_{x\in[a,b]}|f(x)-p(x)|\leqslant\frac{1}{(n+1)!}||f^{(n+1)}||\max_{x\in[a,b]}\prod_{k=0}^n|(x-x_k)|$$
\end{document}