\documentclass[11pt]{article}
\usepackage[utf8]{inputenc}	

%common math and LaTeX packages
\usepackage{amsmath,amsthm,amsfonts,amssymb,amscd}
\usepackage{multirow,booktabs}
\usepackage[table]{xcolor}
\usepackage{multirow}
\usepackage{fullpage}
\usepackage{lastpage}
\usepackage{enumitem}
\usepackage{fancyhdr}
\usepackage{mathrsfs}
\usepackage{wrapfig}
\usepackage{setspace}
\usepackage{calc}
\usepackage{multicol}
\usepackage{cancel}
\usepackage[retainorgcmds]{IEEEtrantools}
\usepackage[margin=1in]{geometry}
\usepackage{amsmath}
\newlength{\tabcont}
\setlength{\parindent}{0.0in}
\setlength{\parskip}{0.0in}
\usepackage{empheq}

%shaded environment for important equations/notes
\usepackage{mdframed}
\colorlet{shaded}{blue!15}
\colorlet{shadedtext}{black}
\newenvironment{shaded}
   {
     \raggedright
     \color{shadedtext}%
   }{}
\surroundwithmdframed[
   hidealllines=true,
   backgroundcolor=shaded
]{shaded}

%page geometry definitions
\usepackage[most]{tcolorbox}
\usepackage{xcolor}
\parindent 0in
\parskip 6pt
\geometry{margin=1in, headsep=0.25in}

%custom theorem definitions
\theoremstyle{definition}
\newtheorem{innercustomgeneric}{\customgenericname}
\providecommand{\customgenericname}{}
\newcommand{\newcustomtheorem}[2]{%
  \newenvironment{#1}[1]
  {%
   \renewcommand\customgenericname{#2}%
   \renewcommand\theinnercustomgeneric{##1}%
   \innercustomgeneric
  }
  {\endinnercustomgeneric}
}
\newcustomtheorem{thm}{Theorem}
\newcustomtheorem{lem}{Lemma}
\newcustomtheorem{defn}{Definition}
\newcustomtheorem{prop}{Proposition}
\newcustomtheorem{exer}{Exercise}
\newcustomtheorem{note}{Note}
\renewcommand{\qedsymbol}{$\blacksquare$}

\let\a\alpha
\let\b\beta
\let\g\gamma
\let\e\varepsilon
\let\t\theta
\newcommand{\R}{\mathbb{R}}
\newcommand{\Q}{\mathbb{Q}}
\newcommand{\Z}{\mathbb{Z}}
\newcommand{\N}{\mathbb{N}}
\newcommand{\PP}{\mathcal{P}}
\newcommand{\C}{\mathcal{C}}
\newcommand{\Lagr}{\mathcal{L}}

\begin{document}

%document header
\begin{center}
{\LARGE \bf ME 202 - Strength of Materials}\\
{Instructor: \textit{Prof. Salil Kulkarni}}\\
Last updated \today \\~\\
{\large \bf Om Prabhu}\\
Undergraduate, Mechanical Engineering\\
Indian Institute of Technology Bombay\\~\\
\textsc{Disclaimer}
\end{center}
\vspace{-6pt}

This document is a compilation of the notes I made while taking the course ME 202 (Strength of Materials) in my 4$^{\text{th}}$ semester at IIT Bombay. Even though I try to discuss as much theory as possible, this is not a substitute to any formal teaching material on the subject.

There will probably be many instances where I use certain common symbols without explicitly mentioning what they mean. It is to be assumed that they carry their usual meanings.

If you have any suggestions and/or spot any errors, you know where to contact me.

\hrulefill
\tableofcontents
\hrulefill
\pagebreak
\section{Torsion of Circular Shafts}
\begin{center}
\begin{tabular}{p{0.1\textwidth}p{0.3\textwidth}p{0.1\textwidth}p{0.2\textwidth}l}
\cline{1-4}
\multicolumn{1}{|l|}{rod}   & \multicolumn{1}{l|}{axial loading}            & \multicolumn{1}{l|}{\multirow{2}{*}{frame}} & \multicolumn{1}{l|}{\multirow{2}{*}{\begin{tabular}[c]{@{}l@{}}axial as well as\\ shear loading\end{tabular}}} &  \\ \cline{1-2}
\multicolumn{1}{|l|}{beam}  & \multicolumn{1}{l|}{transverse/shear loading} & \multicolumn{1}{l|}{}                       & \multicolumn{1}{l|}{}                                                                                          &  \\ \cline{1-4}
\multicolumn{1}{|l|}{shaft} & \multicolumn{1}{l|}{torsional loading}        &                                             &                                                                                                                &  \\ \cline{1-2}
                            &                                               &                                             &                                                                                                                & 
\end{tabular}
\end{center}
\vspace{-12mm}
\begin{itemize}
	\itemsep-0.25em
	\item[$-$] torque: causes twist or \textit{torsion} in a machine element
	\item[$-$] shaft: transmits rotary motion from one location to another
\end{itemize}
\vspace{-7mm}
\begin{enumerate}[label=\textbf{\roman*)}]
	\item \textbf{Internal Resisting Torque} (method of sections)
\end{enumerate}
\vspace{-8mm}

\begin{wrapfigure}[6]{L}{0.42\textwidth}
\includegraphics[width=0.4\textwidth]{section_analysis.JPG}
\end{wrapfigure}

\hspace{0.5\textwidth}
\vspace{-1.5mm}

FBDs at different sections:
\vspace{2.5mm}

\includegraphics[width=0.5\textwidth]{section_FBD.JPG}

$\sum M_{AB}=0\implies -250+T_{AB}=0$

$\sum M_{BC}=0\implies -250+75+T_{BC}=0$

$\sum M_{CD}=0\implies -250+75+325+T_{CD}=0$
\vspace{-10mm}
\begin{wrapfigure}[6]{R}{0.2\textwidth}
\centering \includegraphics[width=0.2\textwidth]{torque_diag.JPG}
\end{wrapfigure}

\vspace{10mm}

The results can be shown using a torque diagram.

Direction of torque is decided using right hand thumb rule, i.e. thumb along +ve z-direction and direction of curling of fingers corresponds to +ve torque. 

To define the origin, consider the element you come across first when you travel along that direction (A, in this case).

If external torque varies with $z$, take a section at an arbitrary $z$ and find internal resisting torque as a function of $z$, i.e. $T(z)$.
\vspace{-2.5mm}

\begin{enumerate}[label=\textbf{\roman*)}]
	\setcounter{enumi}{1}
	\item \textbf{Some Observations}
\end{enumerate}
\vspace{-8mm}

\begin{wrapfigure}[8]{L}{0.27\textwidth}
\includegraphics[width=0.27\textwidth]{torsion_obs.JPG}
\end{wrapfigure}

\hspace{0.5\textwidth}
\vspace{-7mm}

\begin{itemize}
\itemsep0em
	\item[$-$] each circular cut remains a circle
	\item[$-$] longitudinal lines deform helically \& intersect the circles at equal angles
	\item[$-$] cross-sections of shaft ends remain flat
	\item[$-$] radial lines on the flat ends remain straight
\end{itemize}
\vspace{-3mm}

Non-circular shafts often undergo a phenomenon called warping (a topic for a later time).
\begin{enumerate}[label=\textbf{\roman*)}]
	\setcounter{enumi}{2}
	\item \textbf{Shear Strain} $\g$
\end{enumerate}
\vspace{-6mm}

\begin{center}
\includegraphics[width=0.35\textwidth]{strain_1.JPG}\hspace{0.1\textwidth}\includegraphics[width=0.3\textwidth]{strain_2.JPG}
\end{center}
\newpage
Notice how the element is initially rectangular and later gets distorted. This gives rise to a shear strain and as a result, shear stress. As a kinematic assumption, the cross-sections remain planar and rotate rigidly under the load.
$$\g=\tan\psi_{z\theta}=\frac{r\phi(z+\Delta z)-r\phi(z)}{\Delta z}\implies \boxed{\g=r\frac{\text{d}\phi}{\text{d}z}}$$
$$\therefore \g_{max}=R\frac{\text{d}\phi}{\text{d}z}\implies \g=\left(\frac{r}{R}\right)\g_{max}$$
\vspace{-6mm}
\begin{enumerate}[label=\textbf{\roman*)}]
	\setcounter{enumi}{3}
	\item \textbf{Shear Stress  $\tau$ and the Torsion Formula}
\end{enumerate}
\vspace{-10mm}

\begin{center}
\includegraphics[width=0.35\textwidth]{torsion_formula1.JPG}\hspace{0.1\textwidth}\includegraphics[width=0.25\textwidth]{torsion_formula2.JPG}
\end{center}
\vspace{-3mm}
$$T(z+\Delta z)+t(z)\Delta z-T(z)=0\implies \frac{\text{d}T}{\text{d}z}+t(z)=0$$
The moment due to the shear stresses must equal the external torque $T(z)$. Thus $\displaystyle T(z)=\int_{A(z)}\tau r\text{d}A$. Also since $\tau$ \& $\g$ satisfy the constitutive relation $\boxed{\tau=G\g}$, thus $\displaystyle \g=\left(\frac{r}{R}\right)\g_{max}\implies \tau=\left(\frac{r}{R}\right)\tau_{max}$.
$$\therefore T=\int_A\tau r\text{d}A=\int_A \tau_{max}\frac{r^2}{R}\text{d}A=\frac{\tau_{max}}{R}\int_Ar^2\text{d}A\implies \boxed{\tau_{max}=\frac{TR}{J}\implies \tau=\frac{Tr}{J}}$$
The polar moment of inertia $\displaystyle J=\int_Ar^2\text{d}A=\left\lbrace\begin{array}{ll}
\dfrac{\pi R^4}{2}&\text{for solid cross-sections}\\
\dfrac{\pi (r_0^4-r_i^4)}{2}&\text{for annular cross-sections}\\
\end{array}\right.$

Due to the complementary properties of the shear stress, an associated shear stress is also developed in the plane parallel to the $z$-axis.
\vspace{-2mm}
\begin{enumerate}[label=\textbf{\roman*)}]
	\setcounter{enumi}{4}
	\item \textbf{Angle of Twist}
\end{enumerate}

\end{document}